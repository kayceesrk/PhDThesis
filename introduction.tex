\chapter{Introduction}

In recent years, there has been a widespread adoption of both multicore and
cloud computing. Multicore processors have become the norm in mobile, desktop
and enterprise computing, with an increasing number of cores being fitted on a
chip with every successive generation. Cloud computing has paved the way for
companies to rent farms of such multicore processors on a pay-per-use basis,
with the ability to dynamically scale on demand. Indeed, many real-world
services for communication, governance, commerce, education, entertainment,
etc., are routinely exposed as a web-service that runs in third-party cloud
compute platforms such as Windows Azure~\cite{Azure} and Amazon’s
EC2~\cite{Ec2}. These services tend to be concurrently accessed by millions of
users, increasingly through multicore-capable mobile and desktop devices.

\noindent \textbf{Strong consistency.} The holy grail of programming such
massively parallel systems is to achieve good scalability without falling prey
to the usual pitfalls of concurrency such as data races, deadlocks and
atomicity violations~\cite{Lu2008}. Traditionally, programmers have relied on
the underlying hardware or storage infrastructure providing a semblance of a
single memory image, shared between all of the concurrent tasks. Operations
from each task appear to be applied to the shared memory in the order in which
they appear locally in each task, and operations from different tasks are
interleaved in some total order. Such a system is said to provide \emph{strong
memory consistency}. Strong consistency is a natural extension of uniprocessor
memory model to a multiprocessor setting. While this strong consistency does
not completely eliminate the possibility of concurrency bugs, it certainly
simplifies reasoning about the behavior concurrent programs.

\noindent \textbf{Implications of programming model.} Our definition of strong
consistency applies equally to the two popular paradigms of concurrent program
design, \emph{shared memory} and \emph{message-passing}, differentiated by the
way in which the concurrent threads interact with each other. In the shared
memory paradigm, threads interact by updating and inspecting shared memory
locations, whereas under the message-passing paradigm, threads interact by
exchanging messages. For this discussion, let us assume that the shared memory
paradigm is realized through read and write primitive to named memory
locations, and message-passing paradigm is captured by asynchronous send and
blocking receive primitives on named point-to-point channels. Other
message-passing paradigms such as synchronous communication, Erlang-style
mailboxes, thread-addressed messages can be implemented on top of
point-to-point asynchronous message passing model.

Under strongly consistent shared memory, a thread performing a read will
witness the latest write to the same memory location by any thread. Under
strongly consistent message-passing, when a thread performs a sends a value $v$
on an empty channel $c$, the sent value $v$ is available to be consumed by
every thread that subsequently performs a receive. Subsequently, when a receive
operation consumes the sent value $v$, the act of consumption is witnessed by
every thread, and no subsequent thread can consume the same value $v$. Indeed,
semantically shared-memory and message-passing paradigms are simply two sides
of the same coin~\cite{Turon2013,Lauer1979}. This is illustrated by the fact
that one model can easily be implemented using the other. For example,
languages like Haskell~\cite{Haskell}, ConcurrentML~\cite{CML} and
Maticore~\cite{Manticore} implement message-passing paradigms over shared
memory, and popular geo-distributed stores such as Dynamo~\cite{Dynamo},
Cassandra~\cite{Cassandra} and Riak~\cite{Riak} implement shared-memory
paradigm over message passing. Hence, strong memory consistency equally
benefits programmers working under either paradigms. Conversely, and more
importantly, any weaker memory consistency semantics affects both paradigms.

\noindent \textbf{Realizing strong consistency.} Depending upon the target
platform, a variety of mechanisms have been proposed to achieve strong
consistency. Shared memory multicore processors designed for mainstream
computing markets tend to have hierarchical memory organization, with private
and shared multi-level caches, and utilize a hardware protocol for keeping the
caches coherent~\cite{Martin2012}. Coherence can be viewed as a mechanism that
transmits a write to a memory location to all the cache copies of the same
location. Typically, each cache line has meta-data attached to it which
indicate whether the local cacheline is invalid, shared or modified. When a
memory location corresponding to a shared cache line is updated, coherence
mechanism invalidates all other remote cache lines that also refer to the same
memory location. A core accessing an invalid cacheline has to fetch the latest
version, which is termed as cache miss.

In a distributed setting, techniques such as atomic broadcast~\cite{},
consensus~\cite{}, distributed transactions~\cite{}, and distributed locking
services~\cite{} are widely used in practice to provide strong consistency.
These mechanisms abstract the underlying complexity of concurrent programming,
and expose a simpler programming model to the developers. For example, models
such as sequential consistency~\cite{Lam79}, linearizability~\cite{} and
serializability~\cite{} are widely used in the construction of concurrent
programs.

\noindent \textbf{Cost of strong consistency.} Despite the simplicity of strong
consistency, with increasing scale, providing strong consistency guarantees is
an increasingly difficult endeavour. Already, for performance reasons, modern
optimizing compilers and multicore processors reorder code in ways that are not
observable by sequential code, but are very much observable under concurrent
execution~\cite{Demange2013,Sewell2010,Sarkar2011,Batty2011}. Hence, the
semblance of strong consistency is broken. However, the harware memory models
do provide coherence, and the language memory models ensure sequential
consistency for programs that do not involve data races.

On the other hand, the complexity and power requirements for hardware support
for cache coherence increases with increasing number of
cores~\cite{Kavadis2010}. The scalability of hardware cache coherence
mechanisms is mainly hindered by the scalability of coherence hardware, the
storage requirements for cache metadata, and the effort to implement and verify
complex coherence protocols. While there are indeed attempts to reduce the cost
of cache coherence hardware on manycore systems~\cite{Martin2012}, hardware
vendors increasingly opt for non cache coherent architectures. Graphics
processing units (GPUs)~\cite{Luebke2004}, the Intel Single-chip Cloud Computer
(SCC)~\cite{Mattson2010}, the Cell BE processor~\cite{Kahle2005}, and the
Runnemede prototype~\cite{Carter2013} are representative examples of non cache
coherent architectures.

Applications that rely upon strong consistency is a distributed setting have to
pay the cost of reduced \emph{availability} in the presence of network
partitions and high \emph{latency}. In particular, Brewer's well-known CAP
theorem~\cite{Brewer2000,Gilbert2002} states that a distributed system cannot
simultaneously provide strong consistency, be available to updates, and
tolerate network partitions. Since network partitions are unavoidable, and
web-services running on geo-distributed systems focus on providing always-on
experience, application developers unfortunately have to give up the advantages
offered by strong consistency. Moreover, techniques for achieving strong
consistency~\cite{}, require \emph{coordination} between the nodes in the
distributed system. In a geo-distributed setting, where inter-node latencies
are in the order of hundreds of milliseconds, the latency hit associated with
strong consistency is unacceptable. Moreover, coordination between nodes in a
geo-distributed setting while processing client requests defeats the whole
purpose of geo-distribution, which is to minimize latency by serving clients
from the closest data center.

\noindent \textbf{Challenges of weak consistency.} In response to these
concerns, scalable compute platforms eschew strong consistency, and instead
rely only on weaker consistency guarantees. Without strong consistency, the
programmer gets to see that there is no longer a coherent shared memory
abstraction, but instead a collection of \emph{coherence domains} between which
updates are lazily exchanged. The onus now falls on the programmer to ensure
that the application meets is correctness requirements.

On non cache coherent multicore architectures, the programmer must explicitly
perform communication actions between local address spaces through message
passing or direct memory access (DMA). On architectures such as Intel SCC and
Runnemede, which provide explicit instructions to invalidate and flush caches,
the programmer must ensure that the cache control instructions are correctly
issued at appropriate junctures in order to maintain a coherent view of the
shared memory. Understandably, this process is notoriously difficult to get
right.

-- This thesis.

%% * One way - strong consistency; simple abstraction, but complicated underneath.
%%	 + Discuss the implication of shared memory multicore processors and distributed systems
%% * Modern multicore and distributed systems eschew strong consistency in
%%	 favor of weak memory consistency.
%%	 + Implications - (1) absence of coherence (2) asynchrony and (3) eventual consistency
%% * Dealing with this is not easy; provide evidence (TODO: references)

