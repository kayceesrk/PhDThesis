\chapter{Introduction}

In recent years, there has been a widespread adoption of both multicore and
cloud computing. Multicore processors have become the norm in mobile, desktop
and enterprise computing, with an increasing number of cores being fitted on a
chip with every successive generation. Cloud computing has paved the way for
companies to rent farms of such multicore processors on a pay-per-use basis,
with the ability to dynamically scale on demand. Indeed, many real-world
services for communication, governance, commerce, education, entertainment,
etc., are routinely exposed as a web-service that runs in third-party cloud
compute platforms such as Windows Azure~\cite{Calder2011} and Amazon’s
EC2~\cite{EC2}. These services tend to be concurrently accessed by millions of
users, increasingly through multicore-capable mobile and desktop devices.

\section{The Problem}

\subsection{Strong Consistency}

The holy grail of programming such massively parallel systems is to achieve
good scalability without falling prey to the usual pitfalls of concurrency such
as data races, deadlocks and atomicity violations~\cite{Lu2008}. Traditionally,
programmers have relied on the underlying hardware or storage infrastructure
providing a semblance of a single memory image, shared between all of the
concurrent tasks. Operations from each task appear to be applied to the shared
memory in the order in which they appear locally in each task, and operations
from different tasks are interleaved in some total order. Such a system is said
to provide \emph{strong memory consistency}. Strong consistency is a natural
extension of uniprocessor memory model to a multiprocessor setting. While this
strong consistency does not completely eliminate the possibility of concurrency
bugs, it certainly simplifies reasoning about the behavior concurrent programs.

\subsection{Implications of the Programming Model}

Our definition of strong consistency applies equally to the two popular
paradigms of concurrent program design, \emph{shared memory} and
\emph{message-passing}, differentiated by the way in which the concurrent
threads interact with each other. In the shared memory paradigm, threads
interact by updating and inspecting shared memory locations, whereas under the
message-passing paradigm, threads interact by exchanging messages. For this
discussion, let us assume that the shared memory paradigm is realized through
read and write primitive to named memory locations, and message-passing
paradigm is captured by asynchronous send and blocking receive primitives on
named point-to-point channels. Other message-passing paradigms such as
synchronous communication, Erlang-style mailboxes, thread-addressed messages
can be implemented on top of point-to-point asynchronous message passing model.

Under strongly consistent shared memory, a thread performing a read will
witness the latest write to the same memory location by any thread. Under
strongly consistent message-passing, when a thread performs a sends a value $v$
on an empty channel $c$, the sent value $v$ is available to be consumed by
every thread that subsequently performs a receive. Subsequently, when a receive
operation consumes the sent value $v$, the act of consumption is witnessed by
every thread, and no subsequent thread can consume the same value $v$. Indeed,
semantically shared-memory and message-passing paradigms are simply two sides
of the same coin~\cite{Turon2013,Lauer1979}. This is illustrated by the fact
that one model can easily be implemented using the other. For example,
languages like Haskell~\cite{haskellsm}, ConcurrentML~\cite{Reppy99} and
Maticore~\cite{Fluet2007} implement message-passing paradigms over shared
memory, and popular geo-distributed stores such as Dynamo~\cite{DeCandia2007},
Cassandra~\cite{Lakshman2010} and Riak~\cite{Riak} implement shared-memory
paradigm over message passing. Hence, strong memory consistency equally
benefits programmers working under either paradigms. Conversely, and more
importantly, any weaker memory consistency semantics affects both paradigms.

\subsection{Realizing Strong Consistency}

Depending upon the target platform, a variety of mechanisms have been proposed
to achieve strong consistency. Shared memory multicore processors designed for
mainstream computing markets tend to have hierarchical memory organization,
with private and shared multi-level caches, and utilize a hardware protocol for
keeping the caches coherent~\cite{Martin2012}. Coherence can be viewed as a
mechanism that transmits a write to a memory location to all the cache copies
of the same location. Typically, each cache line has meta-data attached to it
which indicate whether the local cacheline is invalid, shared or modified. When
a memory location corresponding to a shared cache line is updated, coherence
mechanism invalidates all other remote cache lines that also refer to the same
memory location. A core accessing an invalid cacheline has to fetch the latest
version, which is termed as cache miss.

In a distributed setting, techniques such as atomic
broadcast~\cite{Defago2004}, consensus~\cite{Lamport98}, distributed
transactions~\cite{Traiger1982}, and distributed locking
services~\cite{Burrows2006} are widely used in practice to provide strong
consistency. These mechanisms abstract the underlying complexity of concurrent
programming, and expose a simpler programming model to the developers. For
example, models such as sequential consistency~\cite{Lam79},
linearizability~\cite{Herlihy1990} and serializability~\cite{Serializability}
are widely used in the construction of concurrent programs.

\subsection{Cost of Strong Consistency}

Despite the simplicity of strong consistency, with increasing scale, providing
strong consistency guarantees is an increasingly difficult endeavour. Already,
for performance reasons, modern optimizing compilers and multicore processors
reorder code in ways that are not observable by sequential code, but are very
much observable under concurrent
execution~\cite{Demange2013,Sewell2010,Sarkar2011,Batty2011}. Hence, the
semblance of strong consistency is broken. However, the harware memory models
do provide coherence, and the language memory models ensure sequential
consistency for programs that do not involve data races.

On the other hand, the complexity and power requirements for hardware support
for cache coherence increases with increasing number of
cores~\cite{Kavadias2010}. The scalability of hardware cache coherence
mechanisms is mainly hindered by the scalability of coherence hardware, the
storage requirements for cache metadata, and the effort to implement and verify
complex coherence protocols. While there are indeed attempts to reduce the cost
of cache coherence hardware on manycore systems~\cite{Martin2012}, hardware
vendors increasingly opt for non cache coherent architectures. Graphics
processing units (GPUs)~\cite{Luebke2004}, the Intel Single-chip Cloud Computer
(SCC)~\cite{Mattson2010}, the Cell BE processor~\cite{Kahle2005}, and the
Runnemede prototype~\cite{Carter2013} are representative examples of non cache
coherent architectures.

Applications that rely upon strong consistency is a distributed setting have to
pay the cost of reduced \emph{availability} in the presence of network
partitions and high \emph{latency}. In particular, Brewer's well-known CAP
theorem~\cite{Brewer2000,Brewer2012,Gilbert2002} states that a distributed
system cannot simultaneously provide strong consistency, be available to
updates, and tolerate network partitions. Since network partitions are
unavoidable, and web-services running on geo-distributed systems focus on
providing always-on experience, application developers unfortunately have to
give up the advantages offered by strong consistency. Moreover, techniques for
achieving strong consistency~\cite{Burrows2006, Traiger1982, Lamport98,
Defago2004}, require \emph{coordination} between the nodes in the distributed
system. In a geo-distributed setting, where inter-node latencies are in the
order of hundreds of milliseconds, the latency hit associated with strong
consistency is unacceptable. Moreover, coordination between nodes in a
geo-distributed setting while processing client requests defeats the whole
purpose of geo-distribution, which is to minimize latency by serving clients
from the closest data center.

\subsection{Challenges under Weak Consistency}

In response to these concerns, scalable compute platforms eschew strong
consistency, and instead rely only on weaker consistency guarantees. Without
strong consistency, the programmer gets to see that there is no longer a
coherent shared memory abstraction, but instead a collection of \emph{coherence
domains} between which updates are lazily exchanged. The onus now falls on the
programmer to ensure that the application meets is correctness requirements.

On non cache coherent multicore architectures, the programmer must explicitly
perform communication actions between local address spaces through message
passing or direct memory access (DMA). On architectures such as Intel
SCC~\cite{Mattson2010} and Runnemede~\cite{Carter2013}, which provide explicit
instructions to invalidate and flush caches, the programmer must ensure that
the cache control instructions are correctly issued at appropriate junctures in
order to maintain a coherent view of the shared memory. Any missed cache
invalidations will lead to stale data being read, where as any missed cache
flushes prevents a write from being exposed to other coherence domains.
However, frequent invalidations and flushes lead to poor cache behavior.
Understandably, this process is notoriously difficult to get right.

A geo-distributed store, where an object is replicated at multiple sites, is in
essense similar to a non cache coherent architecture. Under weak consistency,
programs operating over geo-distributed stores typically assume that the
replicas of an object will \emph{eventually converge} to the same state. This
behavior is commonly termed as eventual
consistency~\cite{Terry1995,Bailis2013}. Unlike multicore architectures, the
high latency in a geo-distributed setting warrants that the application accept
concurrent conflicting updates in order to remain responsive. The updates are
asynchronously propogated between the sites, and a \emph{deterministic}
conflict resolution procedure ensures that the replicas eventually converge to
the same state. The conflict resolution can either be automatic (such as
last-writer-wins) or, in cases where the automatic resolution is non-trivial or
non-existent, may involve manual intervention.

It is important to point out that eventual consistency only guarantees that the
replicas will "eventually" converge to a same state, but does not provide any
additional guarantees with respect to recency or causality of the operations.
Hence, with two successive reads to the same object, there is no guarantee that
the second read will see a "newer" version of the object. Worse still, a
session might not see its own writes! These anamolies are reminiscent of the
reorderings that can occur under language and hardware memory
models~\cite{Demange2013,Sewell2010}, except that the anamolies in this case
are due to the fact that requests from same session can be serviced by
different replicas.

To address these concerns, several systems~\cite{COPS, Eiger, Walter, Li2012,
HAT} have proposed that provide a lattice of stronger guarantees on demand.
While defining new consistency guarantees and implementing them in a
geo-distributed storage infrastructure is certainly a commendable endeavour,
how does one match the consistency requirements at the application level with
the consistency guarantees offered by the store? How does one ensure that the
composition of consistency guarantees of different operations result in a
sensible behavior? In short, developing correct concurrent applications under
weak consistency requires large programmer effort in order to intricately
reasoning about non-trivial memoty interactions on top an already
non-deterministic programming model.

\section{My Thesis}

In this dissertation, we argue that functional programming language
abstractions can mitigate the complexity of programming weakly consistent
systems. The key idea is that, since consistency issues arise out of shared
state mutation, by controlling and minimizing mutation one can simplify the
problem of programming under weak memory consistency.

The dissertation presents three major contributions, each focussed on
addressing a particular challenge associated with weakly consistent loosely
coupled systems: (1) providing an efficient virtual shared memory abstraction
over non cache coherent architectures by exploiting mutability information, (2)
utilizing composable synchronous message-passing communication as an efficient
abstraction for programming asynchronous distributed systems with the help of
speculative execution, and (3) a mutation-free programming model for eventual
consistency that automates the choice of mapping application-level consistency
requirements to consistency levels offered by the geo-distributed data store.

\section{Contributions}

In this section, we provide a brief overview of the contributions made by this
dissertation.

\subsection{Efficiently Masking the Absence of Cache Coherence}

The first contribution of this thesis is a series of techniques to efficiently
hide the absence of cache coherence on a non cache coherent architecture, and
provide the semblance of a shared coherent global address space with strong
(sequential) consistency. We demonstrate this by designing and implementing
\MMSCC, an extension of \MM~\cite{JFP14} compiler and runtime system targetted
at the 48-core memory-coupled, non cache coherent Intel SCC processor.

Providing virtual shared memory on top of distributed memory architectures is
certainly not a novel endeavour. Typically, non cache coherent architectures
are organized such that each core or a collection of cores share a cache
coherent address space (termed as a "coherence domain"), and utilize explicit
communication or DMA transfers for traffic across coherence domains. Such
virtual memory systems additionally implement all the necessary inter-core
communication operations for scheduling and synchronization. This model has
been used on the Cell BE processor for implementing shared-memory programming
models such as OpenMP~\cite{OBrien2008}, COMIC~\cite{Lee2008},
Sequoia~\cite{Houston2008} and CellSs~\cite{Bellens2006}, and on the Intel SCC
for X10~\cite{Chapman2011} and Shared virtual memory model~\cite{Lankes2012}.
These works typically expose the distribution in the programming model, provide
specialized hooks into the architectural features, or are simply agnostic of
the application level consistency requirements.

Differing from these works our work utilizes the key property of
mostly-functional languages (in our case, Standard ML enriched with concurrent
threads and synchronous message passing), that is \emph{mutation is rare}, to
efficiently realize a virtual memory abstration using just the language runtime
mechanisms. We also identify that non cache coherent architectures provide
several different alternatives for inter-core communication such as on-chip
high-speed message passing interconnect, scalable NoC interconnect for
transferring data directly between memory banks without involing the
processors, and explicit cache control instructions. We aim to allow the same
programs written for cache coherent architectures to efficiently run on non
cache coherent architectures, while transparently mapping the source language
structures and mechanisms on to the architecture's capabilities.

Our initial system design utilizes a split-heap memory manager
design~\cite{Marlow11,Auhagen11,Anderson10}, optimized for the SCC's memory
hierarchy, to obtain a \MM system on the SCC. This design however incorporates
both read and write barriers, and we identify that the cost of read barriers
under \MM programming model is significant. To alleviate this, we design a
novel thread local collector that utilizes ample concurrency in the programming
model as a resource along with a dynamic shape analysis to eliminate the read
barriers. Our final runtime design transparently utilizes SCC's support for
software managed cache coherence and on-die message-passing interconnect to
achieve an efficient implementation under which 99\% of the memory accesses can
potentially be cached.

\subsection{A Prescription for Safely Relaxing Synchrony}

The second contribution of the thesis is \rxcml, an optimistic variant of
Concurrent ML~\cite{Reppy99}. \rxcml utilizes synchronous communication over
first-class channels as an abstraction for programming asynchronous distributed
systems. A mostly functional programming language combined with synchronous
message passing over first-class channels offers an attractive and generic
model for expressing fine-grained concurrency. In particular, an expressive
langauge like ConcurrentML~\cite{Reppy99} composable synchronous events, the
synchronous communication simplifies program reasoning by combining data
transfer and synchronization into a single atomic unit. However, in a
distributed setting, such a programming model becomes unviable due to two
reasons:

\begin{itemize}
\item In a geo-distributed setting, synchronization requires coordination
between nodes, which is at odds with the high inter-node latency.
\item As discussed previously, the point-to-point first-class channel
abstraction requires strong consistency. In particular, the channel abstraction
ensures that values are consumed \emph{exactly-once}, which requires
coordination between nodes that might potentially consume a particular value on
the channel.
\end{itemize}

While switching to an explicit asynchronous communication model between nodes
can improve performance, it complicates inter-node synchronization and
introduces naming issues. No longer can a programmer abstractly reason about a
collection of nodes that might send or receive values on a named channel, but
has to identify, communicate and coordinate with individual nodes.
Additionally, the onus falls on the programmers to handle partial failures and
network partitions. Thus, the loss of synchronous communication abstraction
significantly burdens the programmer.

The key contribution of this work is to utilize synchronous communication as an
abstraction to express programs for high-latency distributed systems, but
\emph{speculatively} discharge the communications asynchronously, and ensure
that the observable behavior mirros that of the original synchronous program.
The key discovery is that the necessary and sufficient condition for divergent
behavior (mis-speculation) is the presence of happens-before cycle in the
depen- dence relation between communication actions. We prove this theorem over
an axiomatic formulation that precisely captures the semantics of speculative
execution. Utilizing this idea, we build an optimistic concurrency control
mechanism for concurrent ML programs, on top of MultiMLton, capable of running
in compute clouds. The implementation uses a novel un-coordinated
checkpoint-recovery mechanism to detect and remediate mis-speculations. Our
experiments on Amazon EC2 validate our thesis that this technique is quite
useful in practice.

\subsection{Declarative Programming over Eventually Consistent Data Stores}

The final contribution of this thesis addresses two related challenges when
programming under eventual consistency on top of geo-distributed stores:

\begin{itemize}
\item How do you describe practical and scalable eventually consistent data types?
\item How do you map the application level consistency properties automatically
to the most efficient of the consistency levels provided by the store?
\end{itemize}

Let us expand on the challenges associated with each of these goals.

Typically, commercial geo-distributed stores such as DynamoDB~\cite{DynamoDB},
Cassandra~\cite{Lakshman2010}, Riak~\cite{Riak} provide a data model that is
reminiscent distributed maps. The key-value pair is usually treated as
registers, with a default last-write-wins (LWW) conflict resolution policy.
Since a LWW register is not suitable for every use case, a small collection of
convergent data types such as counters and sets~\cite{SSS} are also provided.
Often the programmer has to coerec the problem at hand, which might naturally
be expressed as operations over a particular abstract datatype into the ones
that are supported by the store. Unlike a concurrent program written for shared
memory multicore processor, the operation on low-level convergent replicated
data types cannot be composed together well; with no practical consistency
control mechanisms such as fences and locks, the programmer has but to reason
about the intricate weak consistency behaviors between composed operation.
Often, without the necessary abstractions, it is impossible to achieve the
desired semantics, and hence, the application ends up exposing the weak
consistency behavior to the user.

In addition, although the store might provide stronger consistency guaratnees,
the lack of precise description of these guarantees, and the inherent
difficulty in mapping application-level consistency requirements to store-level
guarnatees leads to subtle weak consistency bugs. Although there has been
progress on the theoretical front to address the concern of reason about
concurrent programs on eventually consistent stores~\cite{Burckhardt2014},
realization of these techniques on full-fledged commercial store
implementations has not yet come by. Thus, the lack of a suitable programming
model for practical replicated data types hinders software development for
eventually consistent systems.

To address these issues, we present \quelea programming system for
declaratively programming eventually consistent systems. Inspired by
operation-based convergent replicated data types, data types in quelea are
defined in terms of its interfaces, and the effects that an operation has on a
data type. Importantly, the state of an object is simply the set of all effects
performed on this object. Every operation performs a \emph{fold} over this set,
and might optionally produce a new effect. The effects performed at a
particular replica is asynchronously transmitted to other replicas. Since each
operation witnesses all the effects, concurrent or otherwise, performed on the
object so far, semantically conflicting operations can be resolved
deterministically. As we will see, this particular abstraction is powerful
enough to describe complex real-world scenarios including twitter-like
microblogging service and an ebay-like auction site.

Implementing and maintaining a robust, scalable geo-distributed store is a
significant undertaking. Indeed, concerns such as liveness, replication,
durability and failure handling must be handled by any realistic distributed
store implementation, but are orthogonal to the consistency related safety
properties that we aim to address in this work. Instead or replicating the
massive engineering effort and in the process introducing subtle concurrency
and scalability issues, we realize \quelea as a shim layer on top of the
industrial strength data store, Cassandra~\cite{Lakshman2010}. This separation
of concerns allows the \quelea programming model to be ported to other
distributed stores as well.

In addition to the datatype description language, \quelea supports a
\emph{contract} language for describing the application-level consistency
properties. The contract language is used to express valid concurrent
executions utilizing a particular replicated data type, over a small corpus of
primitive relations, capturing properties such as visibility and session order.
The executions described are similar to the the axiomatic description of
relaxed memory models~\cite{Demange2013,Burckhardt2014}, declaratively
capturing the well-formed behaviors in the program. Given a set of store-level
consistency guarantees, also expressed in the same contract language, we
\emph{statically} map each datatype operation to one of the store-level
consistency properties.

Finally, our implementation of the \quelea programming model not only supports
primitive operations, but also a series of \emph{coordination-free
transactions}. Similar to basic operations, we utilize the same contract
language to map the user-defined transactions to one of the store-specific
transaction isolation levels. The thesis illustrates that a mutation-free
programming model for eventually consistent stores not only enables expressive
declarative reasoning, but is also practically achievable on top of
industrial-strength geo-distributed stores.

\section{Road Map}

The rest of the dissertation is organized as follows. Chapter~\ref{chap:mm}
describes the \MM programming model and runtime system, which serves as the
exploration vehicle for \MMSCC and \rxcml. Chapter~\ref{chap:aneris} presents
\MMSCC, the port of \MM to the Intel SCC platform that provides a cache
coherent shared memory abstraction for a concurrent extension of Standard ML.
Chapter~\ref{chap:rxcml} presents \rxcml, an optimistic variant of Concurrent
ML~\cite{Reppy99} for distributed systems. Chapter~\ref{chap:quelea} presents
\quelea, a programming system for eventually consistent geo-distributed stores.
Related work is presented at the end of each chapter. Additional related work
that is relevant to the future direction of this research is given in
chapter~\ref{chap:conclusion}, along with concluding remarks.
