\chapter{Concluding Remarks and Future Work}
\label{chap:conclusion}

A strongly consistent view of data, which enables the programmer to treat
parallel and distributed architectures as a centralized system, is at odds with
practical concerns such as availability, coherence, latency, and partial
failures. Hence, modern multicore and distributed systems only provide weak
consistency guarantees, belying the semblance of a centralized system, which
complicates concurrent programming. In this dissertation, we presented three
novel techniques for programming under weak consistency. \MMSCC provides a
coherent and managed shared memory for programmers on the non-cache coherent
Intel SCC processor. \rxcml enables synchronous communication to be utilized as
an abstraction over asynchronous distributed systems. \quelea permits
declarative reasoning about consistency guarantees for programs over eventually
consistent data stores.

In this chapter, we present the future work. This discussion is split based on
the three contributions.

\section{\MMSCC}

The main hindrance to scalability of our \MMSCC collector is the stop-the-world
nature of the shared heap collection. While shared heap collections are
infrequent when compared to local collections, the pause time for shared heap
collections reaches almost one second due to (1) the uncached nature of the
collection and (2) the cost of synchronizing all the cores on a barrier. A
natural extension to address this issue is to make the shared heap collection
concurrent similar to the design by Doligez et al.~\cite{Doligez93}. In a
concurrent shared heap collection, the cores no longer need to synchronize on a
global barrier, and we could envision allocating a few of the available cores
specifically for concurrent shared heap collection.

Our globalization strategy lifts the \emph{entire} transitive closure of the
globalized object to the shared heap. We have observed that this strategy
although simple to implement, globalizes far more memory than is actually
shared between the cores. This phenomenon has also been observed by Marlow et
al.~\cite{Marlow11} in their local collector design for Haskell. We can
envision a strategy where only portions of the transitive closure are
globalized, with further globalization on demand. In this design, we will have
pointers from shared heap into local heaps (breaking the heap invariant). We
can treat such pointers similar to remembered
stacks (Section~\ref{sec:aneris_rem_stack}), adding the pointers into the local heap
into a remembered set, so that they can be traced during local GCs. Moreover,
the abundance of concurrency in our programming model can mask the latency
associated with on-demand globalization, which involves cross-core
communication.

Finally, while our GC design is geared for circumventing the absence of cache
coherence, we get the added benefit of reduced pause times since each local
heap is collected independently. However, our local heap collections are indeed
optimized for throughput and optimal memory utilization. If latency is indeed
the desired metric, we can envision concurrent and incremental collection for
the local heaps. In particular, the independence of collection in the local
heaps in \MMSCC allows the same execution to utilize latency sensitive GC in a
collection of cores with throughput optimized GC in others. Thus, \MMSCC
design is well-suited for mixed mode applications such as web-browsers, where
both latency and throughput are important for distinct parts of the same
program.

\section{\rxcml}

While \rxcml provides composable synchronous reasoning for asynchronous
distributed systems, the implementation does not address the challenge of
partial failures in such a setting. The key observation we make is that the
dependence graph used for monitoring the correctness of speculative execution
can be persisted to create a checkpoint~\cite{RollbackRecovery} to recover from
failures in a crash-restart mode. Such an implementation is especially useful
in the context of long running data analytics jobs or stateful stream
processing applications.

Currently, \rxcml treats references as side-effecting operations. However, the
techniques used for speculative execution can naturally be extended to
references~\cite{Ziarek10}. In particular, we will treat the reference write as
an effect, and record the old value of the reference written as a node in the
dependence graph. If the execution mis-speculates, apart from restoring the
thread state with saved continuations, we will restore the state of the
references as well.

The \rxcml model can also provide an alternative strategy for enforcing
application-level consistency guarantees for programs on top of eventually
consistent distributed stores. Indeed, distributed stores such as
Bayou~\cite{Bayou} and Google's App Engine datastore utilize speculative
execution to recover stronger consistency guarantees. Equipped with \quelea
style contracts, \rxcml can bring speculative execution for user-defined
replicated data types.

\section{\quelea}

Contracts in \quelea are written by the programmer by mentally translating the
application level consistency specification into visibility constraints on
effects. Ideally, we would like to automatically perform the translation from
database integrity constraints to contracts capturing visibility obligations.
For example, one might wish to express that the balance in a replicated bank
account never drops below zero, which entails the visibility constraint that
withdraw operations must be totally ordered. The task would then be to discover
the \emph{weakest} contract that preserves the invariants. An attractive
approach to solving this problem is to utilize counter-example guided invariant
synthesis~\cite{Cegis} to infer the contracts.

The summarization function for most data types turn out to be straight-forward.
Conway et al.~\cite{edelweiss} describe a program analysis technique to analyze
Bloom programs to automatically derive the garbage collection procedure for
message-passing programs. It would be interesting to explore the applicability
of a similar technique for deriving the summarization function for the RDTs.
The combination of these techniques allow programs for eventually consistent
distributed stores to be expressed in the same way as traditional database
manipulating programs such as SQL or LINQ~\cite{Meijer2011}.

In our current work, we have utilized Cassandra as our backing store. However,
\quelea itself is an abstract model and can be mapped to a variety of backends.
A particularly attractive scenario is utilize \quelea to write programs on top
of non-cache coherent multicore processors like the Intel SCC. Since \quelea
programming model is built for eventually consistent loosely coupled setting,
it can naturally express programs for architectures like SCC. In particular,
each core can operate completely locally, and there is no need for the shared
heap. The same \quelea program can be compiled to a variety of backends
depending upon the deployment platform.
