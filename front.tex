%
%  revised  front.tex  2011-09-02  Mark Senn  http://engineering.purdue.edu/~mark
%  created  front.tex  2003-06-02  Mark Senn  http://engineering.purdue.edu/~mark
%
%  This is ``front matter'' for the thesis.
%
%  Regarding ``References'' below:
%      KEY    MEANING
%      PU     ``A Manual for the Preparation of Graduate Theses'',
%             The Graduate School, Purdue University, 1996.
%      TCMOS  The Chicago Manual of Style, Edition 14.
%      WNNCD  Webster's Ninth New Collegiate Dictionary.
%
%  Lines marked with "%%" may need to be changed.
%

  % Dedication page is optional.
  % A name and often a message in tribute to a person or cause.
  % References: PU 15, WNNCD 332.
\begin{dedication}
To Siva, who always believed it is better to go down trying.
\end{dedication}

  % Acknowledgements page is optional but most theses include
  % a brief statement of apreciation or recognition of special
  % assistance.
  % Reference: PU 16.
\begin{acknowledgments}
  This is the acknowledgments.
\end{acknowledgments}

  % The Table of Contents is required.
  % The Table of Contents will be automatically created for you
  % using information you supply in
  %     \chapter
  %     \section
  %     \subsection
  %     \subsubsection
  % commands.
  % Reference: PU 16.
\tableofcontents

  % If your thesis has tables, a list of tables is required.
  % The List of Tables will be automatically created for you using
  % information you supply in
  %     \begin{table} ... \end{table}
  % environments.
  % Reference: PU 16.
\listoftables

  % If your thesis has figures, a list of figures is required.
  % The List of Figures will be automatically created for you using
  % information you supply in
  %     \begin{figure} ... \end{figure}
  % environments.
  % Reference: PU 16.
\listoffigures

\begin{abbreviations}
	ACML& Asynchronous Concurrent ML\cr
	CML& Concurrent ML\cr
	CSH& Cached Shared Heap\cr
	DMA& Direct memory access\cr
	GC& Garbage-collector\cr
	MPB& Message-passing Buffers\cr
  SCC& Single-chip cloud computer\cr
	SMC& Software-managed cache coherence\cr
	USH& Uncached Shared Heap\cr
\end{abbreviations}

  % Abstract is required.
  % Note that the information for the first paragraph of the output
  % doesn't need to be input here...it is put in automatically from
  % information you supplied earlier using \title, \author, \degree,
  % and \majorprof.
  % Reference: PU 17.
\begin{abstract}

In recent years, there has been a wide-spread adoption of both multicore and
cloud computing. Traditionally, concurrent programmers have relied on the
underlying system providing \emph{strong memory consistency}, where there is a
semblance of concurrent tasks operating over a shared global address space,
with reads to a memory location witnessing the latest write to the same
location. However, providing scalable strong consistency guarantees as the
scale of the system grows is an increasingly difficult endeavour. In a
multicore setting, the increasing complexity and the lack of scalability of
hardware abstractions such as on-chip cache coherence deters scalable strong
consistency. In geo-distributed compute clouds, the latency and availability
concerns in the presence of partial failures prohibit strong consistency.
Hence, modern multicore and cloud computing platforms eschew strong consistency
in favor of weakly consistent memory, meaning that each task has a view of
memory that is incomparable with the other tasks. As a result, programmers on
these platforms must tackle the full complexity of concurrent programming for
an asynchronous distributed system.

This dissertation argues that functional programming language abstractions can
mitigate the complexity of programming weakly consistent systems. It presents
three major contributions, each focussed on addressing a particular challenge
associated with weakly consistent loosely coupled systems: the absence of cache
coherence, asynchrony and eventual consistency. First, it describes \MMSCC, a
parallel extension of MLton Standard ML compiler and runtime for the Intel
Single-chip cloud computer (SCC), and shows how to provide an efficient cache
coherent virtual address space on top of a \emph{non cache coherent} multicore
architecture. Next, it describes \rxcml, a distributed extension of \MM and
shows that, with the help of speculative execution, synchronous communication
can be utilized as an efficient abstraction for programming asynchronous
distributed systems. Finally, it presents the \quelea, a programming system for
\emph{eventually consistent} distributed stores, and shows that the choice of
correct consistency level for replicated data type operations and transactions
can be automated with the help of high-level declarative \emph{contracts}.

\end{abstract}
